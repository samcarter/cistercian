% \iffalse meta-comment
% !TeX TS-program = pdflatex %.dtx | txs:///view-log | txs:///view-pdf 
%<*internal>
\iffalse
%</internal>
%<*internal>
\fi
\def\nameofplainTeX{plain}
\ifx\fmtname\nameofplainTeX\else
  \expandafter\begingroup
\fi
%</internal>
%<*install>
\input docstrip.tex
\keepsilent
\askforoverwritefalse
\preamble

-------------------------------------------------------------------

The cistercian package 
A package for using Cistercian numerals in latex

-------------------------------------------------------------------

\endpreamble
\generate{
  \file{\jobname.sty}{\from{\jobname.dtx}{package}}
}
%</install>
%<install>\endbatchfile
%<*internal>
\generate{
  \file{\jobname.ins}{\from{\jobname.dtx}{install}}
}
\nopreamble\nopostamble
\ifx\fmtname\nameofplainTeX
  \expandafter\endbatchfile
\else
  \expandafter\endgroup
\fi
%</internal>

\ProvidesPackage{cistercian}[2021/04/28 v0.1 Cistercian numerals]

\RequirePackage{tikz}

%<*driver>
\documentclass{ltxdoc}
\usepackage{\jobname} 
\usepackage{parskip}
\usepackage[colorlinks]{hyperref}
\renewcommand{\thepage}{\cistercian[scale=2]{\value{page}}}
\author{names}
\title{The cistercian package}
\begin{document}
\maketitle

The \verb|cistercian| package is meant for displaying Cistercian numerals in \LaTeX{}. 
It was inspired by the Numberphile video \href{https://www.youtube.com/watch?v=9p55Qgt7Ciw}{The Forgotten Number System}. 

Cistercian numerals are a nearly forgotten number system which was invented in the late mediaevals by Cistercian monks, which at that time were one of the most influential congregations with monasteries spreading over most of Europe.

Cistercian numerals are a base 10 system for natural numbers up to 9999. 
They are formed by a vertical stem for zero \cistercian{0}.
Onto this stem, additional markings are added to denote individual digits, e.g. 1 is \cistercian{1}, 2 is \cistercian{2}, etc. 
The markings for ones are added at the top right, the ones for tens are mirrored to the top left (e.g. 10 is \cistercian{10}), hundreds at the bottom right (e.g. 100 is \cistercian{100}) and thousands at the bottom left (e.g. \cistercian{1000}). 
To form the complete number, the digits are superimposed on top of each other, e.g. 1111 is \cistercian{1111}. 

While arithmetic calculations are hard with such a numeral system, it provides a compact and uniform way to denote natural numbers, making it well suited for year numbers or sequences like page numbers or footnotes. 

Several variants were in use, in this package we are adopting the glyph shapes as as shown in \href{https://upload.wikimedia.org/wikipedia/commons/6/67/Cistercian_digits_%28vertical%29.svg}{this diagram}:

\bigskip
\begin{center}
\makeatletter
\renewcommand{\cistercian@height}{4.5ex}%
\renewcommand{\cistercian@width}{1.5ex}%
\renewcommand{\cistercian@stroke}{0.3ex}%
\makeatother
\begin{tabular}{*{9}{c}}
\cistercian{1} & \cistercian{2} & \cistercian{3} & \cistercian{4} & \cistercian{5} & \cistercian{6} & \cistercian{7} & \cistercian{8} & \cistercian{9} \\
1 & 2 & 3 & 4 & 5 & 6 & 7 & 8 & 9\\
\cistercian{10} & \cistercian{20} & \cistercian{30} & \cistercian{40} & \cistercian{50} & \cistercian{60} & \cistercian{70} & \cistercian{80} & \cistercian{90} \\
10 & 20 & 30 & 40 & 50 & 60 & 70 & 80 & 90\\
\cistercian{100} & \cistercian{200} & \cistercian{300} & \cistercian{400} & \cistercian{500} & \cistercian{600} & \cistercian{700} & \cistercian{800} & \cistercian{900} \\
100 & 200 & 300 & 400 & 500 & 600 & 700 & 800 & 900\\
\cistercian{1000} & \cistercian{2000} & \cistercian{3000} & \cistercian{4000} & \cistercian{5000} & \cistercian{6000} & \cistercian{7000} & \cistercian{8000} & \cistercian{9000} \\
1000 & 2000 & 3000 & 4000 & 5000 & 6000 & 7000 & 8000 & 9000\\
\end{tabular}
\end{center}

The \verb|cistercian| package provides the macro \verb|\cistercian{<number>}| to show a given number as Cistercian numeral. 

Internally the package uses a \verb|tikzpicture| to draw the numeral accepts as optional argument all options which can normally be passed to the \verb|tikzpicture| environment (with one exception for the \verb|line width|, but this can be adjusted seperatly, see below). For exampe to get a coloured and scaled numeral \verb|\cistercian[scale=5,draw=red]{3141}|: 

\bigskip
\begin{center}
\cistercian[scale=5,draw=red]{3141}
\end{center}
\bigskip

By default the package scales the Cistercian numerals to be \verb|1.5ex| tall and \verb|0.5ex| wide, with a stroke width of \verb|0.1ex|. 
With the default computer modern font that is approximately the size of a capital letter. 
The user can change this by redefining the following macros after loading the package:

\begin{verbatim}
\renewcommand{\cistercian@height}{1.5ex}
\renewcommand{\cistercian@width}{0.5ex}
\renewcommand{\cistercian@stroke}{0.1ex}
\end{verbatim}
\end{document}
%</driver>
% \fi
% 
%\begin{macrocode}
\newcommand{\cistercian@height}{1.5ex}%
\newcommand{\cistercian@width}{0.5ex}%
\newcommand{\cistercian@stroke}{0.1ex}%

%% #1: optional arguments to be passed to the tikzpicture
%% #2: number to be displayed
\newcommand{\cistercian}[2][]{%
  \ifnum#2<10000
  \unless\ifnum#2<0
    \begin{tikzpicture}[#1]%
      %
      % store scaling factor
      \pgfgettransformentries{\tmpscaleA}{\tmpscaleB}{\tmpscaleC}{\tmpscaleD}{\tmp}{\tmp}%
      \pgfmathsetmacro{\cistercian@scalingfactor}{sqrt(abs(\tmpscaleA*\tmpscaleD-\tmpscaleB*\tmpscaleC))*sqrt(abs((\pgf@xx/1cm)*(\pgf@yy/1cm)-(\pgf@xy/1cm)*(\pgf@yx/1cm)))}%
      %
      % ones
      \pgfmathparse{int(mod(#2,10)}
      \let\cistercian@ones\pgfmathresult
      %
      % tens
      \pgfmathparse{int((mod(#2,100)-mod(#2,10))/10)}
      \let\cistercian@tens\pgfmathresult
      %
      % hundrets        
      \pgfmathparse{int((mod(#2,1000)-mod(#2,100))/100)}
      \let\cistercian@hundrets\pgfmathresult
      %
      % thousands    
      \pgfmathparse{int((mod(#2,10000)-mod(#2,1000))/1000)}
      \let\cistercian@thousands\pgfmathresult
      %
      % superimposing the digts by mirroring
      \cistercian@digit{\cistercian@ones}{\cistercian@tens}
      \cistercian@digit[xscale=-1]{\cistercian@tens}{\cistercian@ones}
      \cistercian@digit[yscale=-1,yshift=-\cistercian@height]{\cistercian@hundrets}{\cistercian@thousands}
      \cistercian@digit[xscale=-1,yscale=-1,yshift=-\cistercian@height]{\cistercian@thousands}{\cistercian@hundrets}
    \end{tikzpicture}%   
  \fi\fi%
}

%% #1: optional arguments for mirroring the digit
%% #2: the digit to be printed
%% #3: digit on the other site of the stem (to handle edge cases of line joints)
\newcommand{\cistercian@digit}[3][]{%
  % making sure all numerals have the same width
  \path[#1,line width=\cistercian@scalingfactor*\cistercian@stroke] (0,0) -- (\cistercian@width,0);
  % drawing the stem for zero
  \draw[#1,line width=\cistercian@scalingfactor*\cistercian@stroke] (0,0) -- ++(0,\cistercian@height)
  % adding features depending in the digit
  \ifnum1=#2%
    -- ++(\cistercian@width,0)
  \fi%
  \ifnum2=#2%
    ++(0,-\cistercian@width) -- ++(\cistercian@width,0)
  \fi%
  \ifnum3=#2%
    % several special case to make sure the line joints with the other site of the number look nice
    \ifnum3=#3%
      ++(-\cistercian@width,-\cistercian@width) -- ++(\cistercian@width,\cistercian@width)  
    \fi
    \ifnum1=#3%
      ++(-\cistercian@width,0) -- ++(\cistercian@width,0)
    \fi  
    \ifnum5=#3%
      ++(-\cistercian@width,0) -- ++(\cistercian@width,0)
    \fi     
    \ifnum7=#3%
      ++(-\cistercian@width,0) -- ++(\cistercian@width,0)
    \fi     
    \ifnum9=#3%
      ++(-\cistercian@width,0) -- ++(\cistercian@width,0)
    \fi             
    -- ++(\cistercian@width,-\cistercian@width)
  \fi%
  \ifnum4=#2%
    ++(0ex,-\cistercian@width) -- ++(\cistercian@width,\cistercian@width)
  \fi%  
  \ifnum5=#2%
    -- ++(\cistercian@width,0ex) -- ++(-\cistercian@width,-\cistercian@width)
  \fi%
  \ifnum6=#2%
    ++(\cistercian@width,0ex) -- ++(0ex,-\cistercian@width)
  \fi%
  \ifnum7=#2%
    --  ++(\cistercian@width,0ex) -- ++(0,-\cistercian@width)
  \fi%
  \ifnum8=#2%
    ++(0,-\cistercian@width) -- ++(\cistercian@width,0) -- ++(0,\cistercian@width)
  \fi%  
  \ifnum9=#2%
    --  ++(\cistercian@width,0ex) -- ++(0,-\cistercian@width) -- ++(-\cistercian@width,0)
  \fi%  
  ;
}
%\end{macrocode}