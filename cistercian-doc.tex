%%%%%%%%%%%%%%%%%%%%%%%%%%%%%%%%%%%%%%%%%%%%%%%%%%%%%%%%%%%%%%%%%%%%%%
%
% Documentation for the cistercian package
% A package to display cistercian numerals 
% Maintained by samcarter
%
% Project repository and bug tracker:
% https://github.com/samcarter/cistercian
%
% Released under the LaTeX Project Public License v1.3c or later
% See https://www.latex-project.org/lppl.txt
%
%%%%%%%%%%%%%%%%%%%%%%%%%%%%%%%%%%%%%%%%%%%%%%%%%%%%%%%%%%%%%%%%%%%%%%
% arara: latexmk
\documentclass{scrartcl}

\usepackage[
  themecolor=samdgray,
]{\jobname-settings}

\usepackage{cistercian}
\makeatletter
\renewcommand{\cistercian@height}{1.25ex}
\renewcommand{\cistercian@stroke}{0.15ex}
\makeatother

\makeatletter
\setlength{\footheight}{21pt}
\cfoot{%
  \renewcommand{\cistercian@height}{1.25ex}%
  \renewcommand{\cistercian@stroke}{0.15ex}%
  \renewcommand{\cistercian@width}{0.5ex}%
  \cistercian[scale=2]{\value{page}}
}
\makeatother
\pagestyle{scrheadings}

% meta %%%%%%%%%%%%%%%%%%%%%%%%%%%%%%%%%%%%%%%%%%%%%%%%%%%%%%%%%%%%%%%
\title{The cistercian package}
\subtitle{Drawing cistercian pieces in \TikZ}
\author{%
  \texorpdfstring{
    \texttt{samcarter}\\
    \url{https://github.com/samcarter/cistercian}\\
%    \url{https://ctan.org/pkg/cistercian}
  }{samcarter}}
\date{Version v0.1 \textendash{} 2025/02/07}
\packagename{cistercian}

\begin{document}
\maketitle
\thispagestyle{scrheadings}

\section{Introduction}

The \saminline|cistercian| package is meant for displaying Cistercian numerals in \LaTeX{}.
It was inspired by the Numberphile video \href{https://www.youtube.com/watch?v=9p55Qgt7Ciw}{The Forgotten Number System}.

\blurb*

\section{Cistercian numerals}

Cistercian numerals are a nearly forgotten number system which was invented in the late medieval by Cistercian monks, which at that time were one of the most influential congregations with monasteries spreading over most of Europe.

Cistercian numerals are a base 10 system for natural numbers up to 9999. 
They are formed by a vertical stem for zero \cistercian{0}.
Onto this stem, additional markings are added to denote individual digits, e.g. 1 is \cistercian{1}, 2 is \cistercian{2}, etc. 
The markings for ones are added at the top right, the ones for tens are mirrored to the top left (e.g. 10 is \cistercian{10}), hundreds at the bottom right (e.g. 100 is \cistercian{100}) and thousands at the bottom left (e.g. \cistercian{1000}). 
To form the complete number, the digits are superimposed on top of each other, e.g. 1111 is \cistercian{1111}. 

While arithmetic calculations are hard with such a numeral system, it provides a compact and uniform way to denote natural numbers, making it well suited for year numbers or sequences like page numbers or footnotes. 

Several variants were in use, in this package we are adopting the glyph shapes as shown in \href{https://upload.wikimedia.org/wikipedia/commons/6/67/Cistercian_digits_%28vertical%29.svg}{this diagram}:

\bigskip
\begin{center}
\makeatletter
\renewcommand{\cistercian@height}{4.5ex}%
\renewcommand{\cistercian@width}{1.5ex}%
\renewcommand{\cistercian@stroke}{0.3ex}%
\makeatother
\begin{tabular}{*{9}{c}}
\cistercian{1} & \cistercian{2} & \cistercian{3} & \cistercian{4} & \cistercian{5} & \cistercian{6} & \cistercian{7} & \cistercian{8} & \cistercian{9} \\
1 & 2 & 3 & 4 & 5 & 6 & 7 & 8 & 9\\
\cistercian{10} & \cistercian{20} & \cistercian{30} & \cistercian{40} & \cistercian{50} & \cistercian{60} & \cistercian{70} & \cistercian{80} & \cistercian{90} \\
10 & 20 & 30 & 40 & 50 & 60 & 70 & 80 & 90\\
\cistercian{100} & \cistercian{200} & \cistercian{300} & \cistercian{400} & \cistercian{500} & \cistercian{600} & \cistercian{700} & \cistercian{800} & \cistercian{900} \\
100 & 200 & 300 & 400 & 500 & 600 & 700 & 800 & 900\\
\cistercian{1000} & \cistercian{2000} & \cistercian{3000} & \cistercian{4000} & \cistercian{5000} & \cistercian{6000} & \cistercian{7000} & \cistercian{8000} & \cistercian{9000} \\
1000 & 2000 & 3000 & 4000 & 5000 & 6000 & 7000 & 8000 & 9000\\
\end{tabular}
\end{center}

\section{Package usage}

The \verb|cistercian| package provides the macro \verb|\cistercian{<number>}| to show a given number as Cistercian numeral. 

Internally the package uses a \verb|tikzpicture| to draw the numeral accepts as optional argument all options which can normally be passed to the \verb|tikzpicture| environment (with one exception for the \verb|line width|, but this can be adjusted separately, see below). For example to get a coloured and scaled numeral

\begin{tcblisting}{title={Jigsaw piece}}
\cistercian[scale=5,draw=red]{3141}
\end{tcblisting}

\bigskip
\begin{center}
\cistercian[scale=5,draw=red]{3141}
\end{center}
\bigskip

By default the package scales the Cistercian numerals to be \verb|1.5ex| tall and \verb|0.5ex| wide, with a stroke width of \verb|0.1ex|. 
With the default computer modern font that is approximately the size of a capital letter. 
The user can change this by redefining the following macros after loading the package:

\begin{verbatim}
\renewcommand{\cistercian@height}{1.5ex}
\renewcommand{\cistercian@width}{0.5ex}
\renewcommand{\cistercian@stroke}{0.1ex}
\end{verbatim}

User, for whom compilation speed is important and who do not need the possibility to modify the appearance of the numerals via Ti\emph{k}Z options, should have a look at the amazingly fast \href{https://ctan.org/pkg/xistercian}{xistercian} package, which also provides additional styles for the numerals. 

\end{document}